\section{Graph Algorithms}


Welcome to the new episode of PrinceOfPersia presents: Fun with algorithms ;)

You can find all the definitions here in the book "Introduction to graph theory", Douglas.B West. Important graph algorithms :
\subsection{DFS}
The most useful graph algorithms are search algorithms. DFS (Depth First Search) is one of them.

While running DFS, we assign colors to the vertices (initially white). Algorithm itself is really simple :
\begin{verbatim}
dfs (v):
        color[v] = gray
        for u in adj[v]:
                if color[u] == white
                        then dfs(u)
        color[v] = black
\end{verbatim}
Black color here is not used, but you can use it sometimes.

Time complexity : $O(n + m)$.
\subsubsection{DFS tree}
DFS tree is a rooted tree that is built like this :

\begin{verbatim}
let T be a new tree
dfs (v):
        color[v] = gray
        for u in adj[v]:
                if color[u] == white
                        then dfs(u) and par[u] = v (in T)

        color[v] = black
\end{verbatim}
Lemma: There is no cross edges, it means if there is an edge between v and u, then v = par[u] or u = par[v].
\subsubsection{Starting time, finishing time}
Starting time of a vertex is the time we enter it (the order we enter it) and its finishing time is the time we leave it. Calculating these are easy :
\begin{verbatim}
TIME = 0
dfs (v):
        st[v] = TIME ++
        color[v] = gray
        for u in adj[v]:
                if color[u] == white
                        then dfs(u)
        color[v] = black
        ft[v] = TIME // or we can use TIME ++
\end{verbatim}
It is useable in specially data structure problems (convert the tree into an array).

Lemma: If we run dfs(root) in a rooted tree, then v is an ancestor of u if and only if stv ≤ stu ≤ ftu ≤ ftv .

So, given arrays st and ft we can rebuild the tree.
\subsubsection{Finding cut edges}
The code below works properly because the lemma above (first lemma):
\begin{verbatim}
h[root] = 0
par[v] = -1
dfs (v):
        d[v] = h[v]
        color[v] = gray
        for u in adj[v]:
                if color[u] == white
                        then par[u] = v and dfs(u) and d[v] = min(d[v], d[u])
                        if d[u] > h[v]
                                then the edge v-u is a cut edge
                else if u != par[v])
                        then d[v] = min(d[v], h[u])
        color[v] = black
\end{verbatim}
In this code, h[v] =  height of vertex v in the DFS tree and d[v] = min(h[w] where there is at least vertex u in subtree of v in the DFS tree where there is an edge between u and w).
\subsubsection{Finding cut vertices}
The code below works properly because the lemma above (first lemma):
\begin{verbatim}
h[root] = 0
par[v] = -1
dfs (v):
        d[v] = h[v]
        color[v] = gray
        for u in adj[v]:
                if color[u] == white
                        then par[u] = v and dfs(u) and d[v] = min(d[v], d[u])
                        if d[u] >= h[v] and (v != root or number_of_children(v) > 1)
                                then the edge v is a cut vertex
                else if u != par[v])
                        then d[v] = min(d[v], h[u])
        color[v] = black
\end{verbatim}
In this code, h[v] =  height of vertex v in the DFS tree and d[v] = min(h[w] where there is at least vertex u in subtree of v in the DFS tree where there is an edge between u and w).
\subsubsection{Finding Eulerian tours}
It is quite like DFS, with a little change :
\begin{verbatim}
vector E
dfs (v):
        color[v] = gray
        for u in adj[v]:
                erase the edge v-u and dfs(u)
        color[v] = black
        push v at the end of e
\end{verbatim}
e is the answer.
\subsection{BFS}
BFS is another search algorithm (Breadth First Search). It is usually used to calculate the distances from a vertex v to all other vertices in unweighted graphs.

Code :
\begin{verbatim}
        BFS(v):
                for each vertex i
                        do d[i] = inf
                d[v] = 0
                queue q
                q.push(v)
                while q is not empty
                        u = q.front()
                        q.pop()
                        for each w in adj[u]
                                if d[w] == inf
                                        then d[w] = d[u] + 1, q.push(w)
\end{verbatim}
Distance of vertex u from v is d[u].

Time complexity : O(n + m).
\subsubsection{BFS tree}
BFS tree is a rooted tree that is built like this :
\begin{verbatim}
let T be a new tree
        BFS(v):
                for each vertex i
                        do d[i] = inf
                d[v] = 0
                queue q
                q.push(v)
                while q is not empty
                        u = q.front()
                        q.pop()
                        for each w in adj[u]
                                if d[w] == inf
                                        then d[w] = d[u] + 1, q.push(w) and par[w] = u (in T)
\end{verbatim}
\subsection{SCC}
The most useful and fast-coding algorithm for finding SCCs is Kosaraju.

In this algorithm, first of all we run DFS on the graph and sort the vertices in decreasing of their finishing time (we can use a stack).

Then, we start from the vertex with the greatest finishing time, and for each vertex v that is not yet in any SCC, do : for each u that v is reachable by u and u is not yet in any SCC, put it in the SCC of vertex v. The code is quite simple.

\subsection{Shortest path}
Shortest path algorithms are algorithms to find some shortest paths in directed or undirected graphs.
\subsubsection{Dijkstra}
This algorithm is a single source shortest path (from one source to any other vertices). Pay attention that you can't have edges with negative weight.

Pseudo code :
\begin{verbatim}
dijkstra(v) :
        d[i] = inf for each vertex i
        d[v] = 0
        s = new empty set
        while s.size() < n
                x = inf
                u = -1
                for each i in V-s //V is the set of vertices
                        if x >= d[i]
                                then x = d[i], u = i
                insert u into s
                // The process from now is called Relaxing
                for each i in adj[u]
                        d[i] = min(d[i], d[u] + w(u,i))
\end{verbatim}
There are two different implementations for this. Both are useful (C++11).

One) $O(n^2)$
\begin{verbatim}
int mark[MAXN];
void dijkstra(int v){
	fill(d,d + n, inf);
	fill(mark, mark + n, false);
	d[v] = 0;
	int u;
	while(true){
		int x = inf;
		u = -1;
		for(int i = 0;i < n;i ++)
			if(!mark[i] and x >= d[i])
				x = d[i], u = i;
		if(u == -1)	break;
		mark[u] = true;
		for(auto p : adj[u]) //adj[v][i] = pair(vertex, weight)
			if(d[p.first] > d[u] + p.second)
				d[p.first] = d[u] + p.second;
	}
}
\end{verbatim}
Two) 

1) Using std :: set :
\begin{verbatim}
void dijkstra(int v){
	fill(d,d + n, inf);
	d[v] = 0;
	int u;
	set<pair<int,int> > s;
	s.insert({d[v], v});
	while(!s.empty()){
		u = s.begin() -> second;
		s.erase(s.begin());
		for(auto p : adj[u]) //adj[v][i] = pair(vertex, weight)
			if(d[p.first] > d[u] + p.second){
				s.erase({d[p.first], p.first});
				d[p.first] = d[u] + p.second;
				s.insert({d[p.first], p.first});
			}
	}
}
\end{verbatim}
2) Using std :: priority$\_$queue (better):
\begin{verbatim}
bool mark[MAXN];
void dijkstra(int v){
	fill(d,d + n, inf);
	fill(mark, mark + n, false);
	d[v] = 0;
	int u;
	priority_queue<pair<int,int>,vector<pair<int,int> >, less<pair<int,int> > > pq;
	pq.push({d[v], v});
	while(!pq.empty()){
		u = pq.top().second;
		pq.pop();
		if(mark[u])
			continue;
		mark[u] = true;
		for(auto p : adj[u]) //adj[v][i] = pair(vertex, weight)
			if(d[p.first] > d[u] + p.second){
				d[p.first] = d[u] + p.second;
				pq.push({d[p.first], p.first});
			}
	}
}
\end{verbatim}

\subsubsection{Floyd-Warshall}
Floyd-Warshal algorithm is an all-pairs shortest path algorithm using dynamic programming.

It is too simple and undrestandable :
\begin{verbatim}
Floyd-Warshal()
	d[v][u] = inf for each pair (v,u)
	d[v][v] = 0 for each vertex v
	for k = 1 to n
		for i = 1 to n
			for j = 1 to n
                d[i][j] = min(d[i][j], d[i][k] + d[k][j])
\end{verbatim}
Time complexity : $O(n^3)$.
\subsubsection{Bellman-Ford}
Bellman-Ford is an algorithm for single source shortest path where edges can be negative (but if there is a cycle with negative weight, then this problem will be NP).

The main idea is to relax all the edges exactly n - 1 times (read relaxation above in dijkstra). You can prove this algorithm using induction.

If in the n - th step, we relax an edge, then we have a negative cycle (this is if and only if).

Code :
\begin{verbatim}
Bellman-Ford(int v)
	d[i] = inf for each vertex i
	d[v] = 0
	for step = 1 to n
		for all edges like e
			i = e.first // first end
			j = e.second // second end
			w = e.weight
			if d[j] > d[i] + w
				if step == n
					then return "Negative cycle found"
                d[j] = d[i] + w
\end{verbatim}
Time complexity : $O(nm)$.
\subsubsection{SPFA}
SPFA (Shortest Path Faster Algorithm) is a fast and simple algorithm (single source) that its complexity is not calculated yet. But if m = O(n2) it's better to use the first implementation of Dijkstra.

The origin of this algorithm is unknown. It's said that at first Chinese coders used it in programming contests.

Its code looks like the combination of Dijkstra and BFS :
\begin{verbatim}
SPFA(v):
	d[i] = inf for each vertex i
	d[v] = 0
	queue q
	q.push(v)
	while q is not empty
		u = q.front()
		q.pop()
		for each i in adj[u]
			if d[i] > d[u] + w(u,i)
				then d[i] = d[u] + w(u,i)
				if i is not in q
                    then q.push(i)
\end{verbatim}
Time complexity : Unknown!.
\subsection{MST}
MST = Minimum Spanning Tree :) (if you don't know what it is, google it).

Best MST algorithms :
\subsubsection{Kruskal}
In this algorithm, first we sort the edges in ascending order of their weight in an array of edges.

Then in order of the sorted array, we add ech edge if and only if after adding it there won't be any cycle (check it using DSU).

Code :
\begin{verbatim}
Kruskal()
	solve all edges in ascending order of their weight in an array e
	ans = 0
	for i = 1 to m
		v = e.first
		u = e.second
		w = e.weight
		if merge(v,u) // there will be no cycle
            then ans += w
\end{verbatim}
Time complexity : $O(m\log{m})$.
\subsubsection{Prim}
In this approach, we act like Dijkstra. We have a set of vertices S, in each step we add the nearest vertex to S, in S (distance of v from $S = \min_{u \in S}(weight(u,v))$ where weight(i, j) is the weight of the edge from i to j) .

So, pseudo code will be like this:
\begin{verbatim}
Prim()
	S = new empty set
	for i = 1 to n
		d[i] = inf
	while S.size() < n
		x = inf
		v = -1
		for each i in V - S // V is the set of vertices
			if x >= d[v]
				then x = d[v], v = i
		d[v] = 0
		S.insert(v)
		for each u in adj[v]
            do d[u] = min(d[u], w(v,u))
\end{verbatim}
C++ code:
One) $O(n^2)$
\begin{verbatim}
bool mark[MAXN];
void prim(){
	fill(d, d + n, inf);
	fill(mark, mark + n, false);
	int x,v;
	while(true){
		x = inf;
		v = -1;
		for(int i = 0;i < n;i ++)
			if(!mark[i] and x >= d[i])
				x = d[i], v = i;
		if(v == -1)
			break;
		d[v] = 0;
		mark[v] = true;
		for(auto p : adj[v]){ //adj[v][i] = pair(vertex, weight)
			int u = p.first, w = p.second;
			d[u] = min(d[u], w);
		}
	}
}
\end{verbatim}
Two) $O(m\log{n})$
\begin{verbatim}
void prim(){
	fill(d, d + n, inf);
	set<pair<int,int> > s;
	for(int i = 0;i < n;i ++)
		s.insert({d[i],i});
	int v;
	while(!s.empty()){
		v = s.begin() -> second;
		s.erase(s.begin());
		for(auto p : adj[v]){
			int u = p.first, w = p.second;
			if(d[u] > w){
				s.erase({d[u], u});
				d[u] = w;
				s.insert({d[u], u});
			}
		}
	}
}
\end{verbatim}
As Dijkstra you can use std :: priority$\_$queue instead of std :: set.

\subsection{Maximum Flow}
I only wanna put the source code here (EdmondsKarp):
\begin{verbatim}
algorithm EdmondsKarp
    input:
        C[1..n, 1..n] (Capacity matrix)
        E[1..n, 1..?] (Neighbour lists)
        s             (Source)
        t             (Sink)
    output:
        f             (Value of maximum flow)
        F             (A matrix giving a legal flow with the maximum value)
    f := 0 (Initial flow is zero)
    F := array(1..n, 1..n) (Residual capacity from u to v is C[u,v] - F[u,v])
    forever
        m, P := BreadthFirstSearch(C, E, s, t, F)
        if m = 0
            break
        f := f + m
        (Backtrack search, and write flow)
        v := t
        while v ≠ s
            u := P[v]
            F[u,v] := F[u,v] + m
            F[v,u] := F[v,u] - m
            v := u
    return (f, F)
	
	
	
algorithm BreadthFirstSearch
    input:
        C, E, s, t, F
    output:
        M[t]          (Capacity of path found)
        P             (Parent table)
    P := array(1..n)
    for u in 1..n
        P[u] := -1
    P[s] := -2 (make sure source is not rediscovered)
    M := array(1..n) (Capacity of found path to node)
    M[s] := ∞
    Q := queue()
    Q.offer(s)
    while Q.size() > 0
        u := Q.poll()
        for v in E[u]
            (If there is available capacity, and v is not seen before in search)
            if C[u,v] - F[u,v] > 0 and P[v] = -1
                P[v] := u
                M[v] := min(M[u], C[u,v] - F[u,v])
                if v ≠ t
                    Q.offer(v)
                else
                    return M[t], P
    return 0, P
\end{verbatim}
	
EdmondsKarp pseudo code using Adjacency nodes:
\begin{verbatim}
algorithm EdmondsKarp
    input:
        graph (Graph with list of Adjacency nodes with capacities,flow,reverse and destinations)
        s             (Source)
        t             (Sink)
    output:
        flow             (Value of maximum flow)
    flow := 0 (Initial flow to zero)
    q := array(1..n) (Initialize q to graph length)
    while true
        qt := 0            (Variable to iterate over all the corresponding edges for a source)
        q[qt++] := s    (initialize source array)
        pred := array(q.length)    (Initialize predecessor List with the graph length)
        for qh=0;qh < qt && pred[t] == null
            cur := q[qh]
            for (graph[cur]) (Iterate over list of Edges)
                 Edge[] e :=  graph[cur]  (Each edge should be associated with Capacity)
                 if pred[e.t] == null && e.cap > e.f
                    pred[e.t] := e
                    q[qt++] : = e.t
        if pred[t] == null
            break
        int df := MAX VALUE (Initialize to max integer value)
        for u = t; u != s; u = pred[u].s
            df := min(df, pred[u].cap - pred[u].f)
        for u = t; u != s; u = pred[u].s
            pred[u].f  := pred[u].f + df
            pEdge := array(PredEdge)
            pEdge := graph[pred[u].t]
            pEdge[pred[u].rev].f := pEdge[pred[u].rev].f - df;
        flow := flow + df
    return flow
\end{verbatim}
\subsubsection{Dinic's algorithm}
Here is Dinic's algorithm as you wanted.

Input: A network G = ((V, E), c, s, t).

Output: A max s - t flow.

1.set f(e) = 0 for each e in E

2.Construct G$\_$L from G$\_$f of G. if dist(t) == inf, then stop and output f 

3.Find a blocking flow fp in G$\_$L

4.Augment flow f by fp  and go back to step 2.

Time complexity : $O(mm\log{n})$.

Theorem: Maximum flow = minimum cut.

\subsubsection{Maximum Matching in bipartite graphs}
Maximum matching in bipartite graphs is solvable also by maximum flow like below :

Add two vertices S, T to the graph, every edge from X to Y (graph parts) has capacity 1, add an edge from S with capacity 1 to every vertex in X, add an edge from every vertex in Y with capacity 1 to T.

Finally, answer = maximum matching from S to T .

But it can be done really easier using DFS.

As, you know, a bipartite matching is the maximum matching if and only if there is no augmenting path (read Introduction to graph theory).

The code below finds a augmenting path:
\begin{verbatim}
bool dfs(int v){// v is in X, it reaturns true if and only if there is an augmenting path starting from v
	if(mark[v])
		return false;
	mark[v] = true;
	for(auto &u : adj[v])
		if(match[u] == -1 or dfs(match[u])) // match[i] = the vertex i is matched with in the current matching, initially -1
			return match[v] = u, match[u] = v, true;
	return false;
}
\end{verbatim}
An easy way to solve the problem is:
\begin{verbatim}
for(int i = 0;i < n;i ++)if(match[i] == -1){
	memset(mark, false, sizeof mark);
	dfs(i);
}
\end{verbatim}
But there is a faster way:
\begin{verbatim}
while(true){
	memset(mark, false, sizeof mark);
	bool fnd = false;
	for(int i = 0;i < n;i ++) if(match[i] == -1 && !mark[i])
		fnd |= dfs(i);
	if(!fnd)
		break;
}
\end{verbatim}
In both cases, time complexity = $O(nm)$.

\subsection{Trees}
Trees are the most important graphs.

In the last lectures we talked about segment trees on trees and heavy-light decomposition.

\subsubsection{Partial sum on trees}
We can also use partial sum on trees.

Example: Having a rooted tree, each vertex has a value (initially 0), each query gives you numbers v and u (v is an ancestor of u) and asks you to increase the value of all vertices in the path from u to v by 1.

So, we have an array p, and for each query, we increase p[u] by 1 and decrease p[par[v]] by 1. The we run this (like a normal partial sum):
\begin{verbatim}
void dfs(int v){
	for(auto u : adj[v])
		if(u - par[v])
			dfs(u), p[v] += p[u];
}
\end{verbatim}
\subsubsection{DSU on trees}
We can use DSU on a rooted tree (not tree DSUs, DSUs like vectors).

For example, in each node, we have a vector, all nodes in its subtree (this can be used only for offline queries, because we may have to delete it for memory usage).

Here again we use DSU technique, we will have a vector V for every node. When we want to have V[v] we should merge the vectors of its children. I mean if its children are u1, u2, ..., uk where V[u1].size() ≤ V[u2].size() ≤ ... ≤ V[uk].size(), we will put all elements from V[ui] for every 1 ≤ i < k, in V[k] and then, V[v] = V[uk].

Using this trick, time complexity will be .

C++ example (it's a little complicated) :
\begin{verbatim}
typedef vector<int> vi;
vi *V[MAXN];
void dfs(int v, int par = -1){
	int mx = 0, chl = -1;
	for(auto u : adj[v])if(par - u){
		dfs(u,v);
		if(mx < V[u]->size()){
			mx = V[u]->size();
			chl = u;
		}
	}
	for(auto u : adj[v])if(par - u and chl - u){
		for(auto a : *V[u])
			V[chl]->push_back(a);
		delete V[u];
	}
	if(chl + 1)
		V[v] = V[chl];
	else{
		V[v] = new vi;
		V[v]->push_back(v);
	}
}
\end{verbatim}
\subsubsection{LCA}
LCA of two vertices in a rooted tree, is their lowest common ancestor.

There are so many algorithms for this, I will discuss the important ones.

Each algorithm has complexities  < O(f(n)), O(g(n)) > , it means that this algorithm's preprocess is O(f(n)) and answering a query is O(g(n)) .

In all algorithms, h[v] =  height of vertex v.
One) Brute force  < O(n), O(n) > 

The simplest approach. We go up enough to achieve the goal.

Preproccess :
\begin{verbatim}
void dfs(int v,int p = -1){
	if(par + 1)
		h[v] = h[p] + 1;
	par[v] = p;
	for(auto u : adj[v])	if(p - u)
		dfs(u,v);
}
\end{verbatim}
Query :
\begin{verbatim}
int LCA(int v,int u){
	if(v == u)
		return v;
	if(h[v] < h[u])
		swap(v,u);
	return LCA(par[v], u);
}
\end{verbatim}
Two) SQRT decomposition 

I talked about SQRT decomposition in the first lecture.

Here, we will cut the tree into $\sqrt{H}$ (H = height of the tree), starting from 0, k - th of them contains all vertices with h in interval $[k\sqrt{H},(k+1)\sqrt{H}]$.

Also, for each vertex v in k - th piece, we store r[v] that is, its lowest ancestor in the piece number k - 1.

Preprocess:
\begin{verbatim}
void dfs(int v,int p = -1){
	if(par + 1)
		h[v] = h[p] + 1;
	par[v] = p;
	if(h[v] % SQRT == 0)
		r[v] = p;
	else
		r[v] = r[p];
	for(auto u : adj[v])	if(p - u)
		dfs(u,v);
}
\end{verbatim}
Query:
\begin{verbatim}
int LCA(int v,int u){
	if(v == u)
		return v;
	if(h[v] < h[u])
		swap(v,u);
	if(h[v] == h[u])
		return (r[v] == r[u] ? LCA(par[v], par[u]) : LCA(r[v], r[u]));
	if(h[v] - h[u] < SQRT)
		return LCA(par[v], u);
	return LCA(r[v], u);
}
\end{verbatim}
Three) Sparse table <$O(n\log{n})$, $O(1)$>

Let's introduce you an order of tree vertices, haas and I named it Euler order. It is like DFS order, but every time we enter a vertex, we write it's number down (even when we come from a child to this node in DFS).

Code for calculate this :
\begin{verbatim}
vector<int> euler;
void dfs(int v,int p = -1){
	euler.push_back(v);
	for(auto u : adj[v])	if(p - u)
		dfs(u,v), euler.push_back(v);
}
\end{verbatim}
If we have a vector<pair<int,int> > instead of this and push {h[v], v} in the vector, and the first time {h[v], v} is appeared is s[v] and s[v] < s[u] then LCA(v, u) = ($min_{i = s[v]}^{s[u]}euler[i]$).second.

For this propose we can use RMQ problem, and the best algorithm for that, is to use Sparse table.

Four) Something like Sparse table :) <$O(n\log{n})$,$O(\log{n})$>

This is the most useful and simple (among fast algorithms) algorithm.

For each vector v and number i, we store its $2^i$-th ancestor. This can be done in $O(n\log{n})$. Then, for each query, we find the lowest ancestors of them which are in the same height, but different (read the source code for understanding).

Preprocess:
\begin{verbatim}
int par[MAXN][MAXLOG]; // initially all -1
void dfs(int v,int p = -1){
	par[v][0] = p;
	if(p + 1)
		h[v] = h[p] + 1;
	for(int i = 1;i < MAXLOG;i ++)
		if(par[v][i-1] + 1)
			par[v][i] = par[par[v][i-1]][i-1];
	for(auto u : adj[v])	if(p - u)
		dfs(u,v);
}
\end{verbatim}
Query:
\begin{verbatim}
int LCA(int v,int u){
	if(h[v] < h[u])
		swap(v,u);
	for(int i = MAXLOG - 1;i >= 0;i --)
		if(par[v][i] + 1 and h[par[v][i]] >= h[u])
			v = par[v][i];
	// now h[v] = h[u]
	if(v == u)
		return v;
	for(int i = MAXLOG - 1;i >= 0;i --)
		if(par[v][i] - par[u][i])
			v = par[v][i], u = par[u][i];
	return par[v][0];
}
\end{verbatim}

Five) Advance RMQ  < O(n), O(1) > 

In the third approach, we said that LCA can be solved by RMQ.

When you look at the vector euler you see that for each i that 1 ≤ i < euler.size(), |euler[i].first - euler[i + 1].first| = 1.

So, we can convert the euler from its size(we consider its size is n + 1) into a binary sequence of length n (if euler[i].first - euler[i + 1].first = 1 we put 1 otherwise 0).

So, we have to solve the problem on a binary sequence A .

To solve this restricted version of the problem we need to partition A into blocks of size . Let A'[i] be the minimum value for the i - th block in A and B[i] be the position of this minimum value in A. Both A and B are  long. Now, we preprocess A' using the Sparse Table algorithm described in lecture 1. This will take  time and space. After this preprocessing we can make queries that span over several blocks in O(1). It remains now to show how the in-block queries can be made. Note that the length of a block is , which is quite small. Also, note that A is a binary array. The total number of binary arrays of size l is . So, for each binary block of size l we need to lock up in a table P the value for RMQ between every pair of indices. This can be trivially computed in  time and space. To index table P, preprocess the type of each block in A and store it in array . The block type is a binary number obtained by replacing  - 1 with 0 and  + 1 with 1 (as described above).

Now, to answer RMQA(i, j) we have two cases:

i and j are in the same block, so we use the value computed in P and T

i and j are in different blocks, so we compute three values: the minimum from i to the end of i's block using P and T, the minimum of all blocks between i's and j's block using precomputed queries on A' and the minimum from the beginning of j's block to j, again using T and P; finally return the position where the overall minimum is using the three values you just computed.

Six) Tarjan's algorithm O(na(n)) (a(n) is the inverse ackermann function)

Tarjan's algorithm is offline; that is, unlike other lowest common ancestor algorithms, it requires that all pairs of nodes for which the lowest common ancestor is desired must be specified in advance. The simplest version of the algorithm uses the union-find data structure, which unlike other lowest common ancestor data structures can take more than constant time per operation when the number of pairs of nodes is similar in magnitude to the number of nodes. A later refinement by Gabow $\&$ Tarjan (1983) speeds the algorithm up to linear time.

The pseudocode below determines the lowest common ancestor of each pair in P, given the root r of a tree in which the children of node n are in the set n.children. For this offline algorithm, the set P must be specified in advance. It uses the MakeSet, Find, and Union functions of a disjoint-set forest. MakeSet(u) removes u to a singleton set, Find(u) returns the standard representative of the set containing u, and Union(u, v) merges the set containing u with the set containing v. TarjanOLCA(r) is first called on the root r.
\begin{verbatim}
 function TarjanOLCA(u)
     MakeSet(u);
     u.ancestor := u;
     for each v in u.children do
         TarjanOLCA(v);
         Union(u,v);
         Find(u).ancestor := u;
     u.colour := black;
     for each v such that {u,v} in P do
         if v.colour == black
             print "Tarjan's Lowest Common Ancestor of " + u +
                   " and " + v + " is " + Find(v).ancestor + ".";
\end{verbatim}
Each node is initially white, and is colored black after it and all its children have been visited. The lowest common ancestor of the pair {u, v} is available as Find(v).ancestor immediately (and only immediately) after u is colored black, provided v is already black. Otherwise, it will be available later as Find(u).ancestor, immediately after v is colored black.
\begin{verbatim}
 function MakeSet(x)
     x.parent := x
     x.rank   := 0
 
 function Union(x, y)
     xRoot := Find(x)
     yRoot := Find(y)
     if xRoot.rank > yRoot.rank
         yRoot.parent := xRoot
     else if xRoot.rank < yRoot.rank
         xRoot.parent := yRoot
     else if xRoot != yRoot
         yRoot.parent := xRoot
         xRoot.rank := xRoot.rank + 1
  
 function Find(x)
     if x.parent == x
        return x
     else
        x.parent := Find(x.parent)
        return x.parent
\end{verbatim}
